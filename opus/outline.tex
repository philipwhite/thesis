\documentclass[12pt]{article}
\usepackage{easylist}
\usepackage{setspace}
\ListProperties(Progressive=0.25in)
\newcommand{\nb}[1]{\scriptsize \bfseries #1}
\begin{document}
\doublespacing
\begin{easylist}
§ Introduction \nb{Not Begun}
§ Literature Review \nb{Not Begun}
§§ Discuss where others have used automated systems to analyze text.
§§ grammar checkers, Microsoft Research's project for English Learners, etc.
§ Introduction to Corpora \nb{Draft Finished}
§§ List corpora, size, organization, etc.
§§ Give statistics for native and non-native instances
§ Overview of the Experiment Format and the Technologies Used \nb{Partly Done}
§§ Use of the Stanford Parser for parsing and dependencies \nb{Partly Done}
§§§ Stanford Parser
§§§ Stanford Typed Dependencies
§§ Use of Weka for classification \nb{Largely Done}
§§§ C4.5 Classifier \nb{Draft Finished}
§§§ RandomForest Classifier \nb{Not Begun}
§§ The Experiment Format \nb{Not Begun}
§§§ The project consists of a suite of tests
§§§ Each test concerns a particular grammatical point
§§§ Each test operates on the output from the Stanford parser and generates training or testing cases for the Weka classifier.
§§§ The cases consist of one or (usually) more attributes with continuous values.
§§§ The values are generally relative frequencies.
§ Experiment 1 -- Grammatical Relations \nb{Partly Done}
§§ The attributed described.
§§ Explanation of how the attributes are gathered.
§§ Show results
§§ Analyze results linguistically.
§ Experiment 2 -- Verbal Arguments \nb{Draft Finished}
§§ \ldots
§ Experiment 3 -- Verb Forms \nb{Draft Finished}
§§ \ldots
§ These experiments as the basis of a learner's tool. \nb{Partly Done}
§ Conclusion \nb{Not Begun}
\end{easylist}

\end{document}