\message{ !name(main.tex)}
\documentclass{paper}

\begin{document}

\message{ !name(main.tex) !offset(-3) }

\textbf{Memo}

The extensive use of elements from vernacular speech by the earliest authors and inscriptions of the Roman Republic make it clear that the original, unwritten language of the Roman Monarchy was an only partially deducible colloquial form, the predecessor to Vulgar Latin. By the late Roman Republic, a standard, literate form had arisen from the speech of the educated, now referred to as Classical Latin. Vulgar Latin, by contrast, is the name given to the more rapidly changing colloquial language spoken throughout the empire.[5] With the Roman conquest, Latin spread to many Mediterranean regions, and the dialects spoken in these areas, mixed to various degrees with the autochthonous languages, developed into the Romance tongues, including Aragonese, Catalan, Corsican, French, Galician, Italian, Portuguese, Romanian, Romansh, Sardinian, Sicilian, and Spanish.[6] Classical Latin slowly changed with the Decline of the Roman Empire, as education and wealth became ever scarcer. The consequent Medieval Latin, influenced by various Germanic and proto-Romance languages until expurgated by Renaissance scholars, was used as the language of international communication, scholarship and science until well into the 18th century, when it began to be supplanted by vernacular languages. Please see Table ~\ref{poop} for fun.


The extensive use of elements from vernacular speech by the earliest authors and inscriptions of the Roman Republic make it clear that the original, unwritten language of the Roman Monarchy was an only partially deducible colloquial form, the predecessor to Vulgar Latin. By the late Roman Republic, a standard, literate form had arisen from the speech of the educated, now referred to as Classical Latin. Vulgar Latin, by contrast, is the name given to the more rapidly changing colloquial language spoken throughout the empire.[5] With the Roman conquest, Latin spread to many Mediterranean regions, and the dialects spoken in these areas, mixed to various degrees with the autochthonous languages, developed into the Romance tongues, including Aragonese, Catalan, Corsican, French, Galician, Italian, Portuguese, Romanian, Romansh, Sardinian, Sicilian, and Spanish.[6] Classical Latin slowly changed with the Decline of the Roman Empire, as education and wealth became ever scarcer. The consequent Medieval Latin, influenced by various Germanic and proto-Romance languages until expurgated by Renaissance scholars, was used as the language of international communication, scholarship and science until well into the 18th century, when it began to be supplanted by vernacular languages. Please see Table ~\ref{poop} for fun.


The extensive use of elements from vernacular speech by the earliest authors and inscriptions of the Roman Republic make it clear that the original, unwritten language of the Roman Monarchy was an only partially deducible colloquial form, the predecessor to Vulgar Latin. By the late Roman Republic, a standard, literate form had arisen from the speech of the educated, now referred to as Classical Latin. Vulgar Latin, by contrast, is the name given to the more rapidly changing colloquial language spoken throughout the empire.[5] With the Roman conquest, Latin spread to many Mediterranean regions, and the dialects spoken in these areas, mixed to various degrees with the autochthonous languages, developed into the Romance tongues, including Aragonese, Catalan, Corsican, French, Galician, Italian, Portuguese, Romanian, Romansh, Sardinian, Sicilian, and Spanish.[6] Classical Latin slowly changed with the Decline of the Roman Empire, as education and wealth became ever scarcer. The consequent Medieval Latin, influenced by various Germanic and proto-Romance languages until expurgated by Renaissance scholars, was used as the language of international communication, scholarship and science until well into the 18th century, when it began to be supplanted by vernacular languages. Please see Table ~\ref{poop} for fun.


The extensive use of elements from vernacular speech by the earliest authors and inscriptions of the Roman Republic make it clear that the original, unwritten language of the Roman Monarchy was an only partially deducible colloquial form, the predecessor to Vulgar Latin. By the late Roman Republic, a standard, literate form had arisen from the speech of the educated, now referred to as Classical Latin. Vulgar Latin, by contrast, is the name given to the more rapidly changing colloquial language spoken throughout the empire.[5] With the Roman conquest, Latin spread to many Mediterranean regions, and the dialects spoken in these areas, mixed to various degrees with the autochthonous languages, developed into the Romance tongues, including Aragonese, Catalan, Corsican, French, Galician, Italian, Portuguese, Romanian, Romansh, Sardinian, Sicilian, and Spanish.[6] Classical Latin slowly changed with the Decline of the Roman Empire, as education and wealth became ever scarcer. The consequent Medieval Latin, influenced by various Germanic and proto-Romance languages until expurgated by Renaissance scholars, was used as the language of international communication, scholarship and science until well into the 18th century, when it began to be supplanted by vernacular languages. Please see Table ~\ref{poop} for fun.


\begin{equation}
\chi(M_g) = 2-2g
\end{equation}

\begin{table}[here]
  \begin{tabular}{|l|c|c|c|}
  \hline
        & 9.00 -- 10.00 & 10.00 -- 11.00 & 11.00 -- 12.00 \\ \hline 
Monday  & Applicable Mathematics I & Basic Engineering I
        & Engineering Design I \\ \hline
Tuesday & Applicable Mathematics I & Basic Engineering I
        & Engineering Practive I \\ \hline
  \end{tabular}
  \caption{First Year Timetable}
\label{poop}
\end{table}

\end{document}

\message{ !name(main.tex) !offset(-41) }
