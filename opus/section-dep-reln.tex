\documentclass[main.tex]{subfiles}
\begin{document}

\section{Grammatical Relations}

The simplest classification approach considered the relative frequency of different grammatical relations. For this approach, the governor and the dependent of the dependencies were ignored, with only the relation itself being used. 

Each data set instance contained 52 numerical attributes, one for each relation in the Stanford Dependency system. For each attribute \(A_r\) corresponding to the relation \(r\), the corresponding value was \(n_r/n_t\), where \(n_r\) and \(n_t\) were the number of occurrences of the relation \(r\) and the total number of relations in the text, respectively.

A C4.5 decision tree classifier trained on these instances produces the decision tree shown in Algorithm~\ref{alg:reln-dtree}. The full names for the seven relations are shown in Table~\ref{table:reln-abbr}. The following subsections explore the linguistic reasons why these particular relations should be so useful in classifying the texts.

\begin{algorithm}[htbp]
\begin{spacing}{1.0}
\caption{C4.5 decision tree classifier}
{\footnotesize
\begin{algorithmic}
\IF {$complm \leq 0.011635$}
  \IF {$purpcl \leq 0.000856$}
    \IF {$rcmod \leq 0.012254$}
      \STATE $en (34.0)$
    \ELSE %rcmod
      \IF {$prt \leq 0.002113$}
        \STATE $es (4.0/1.0)$
      \ELSE %prt
        \STATE $en (6.0)$
      \ENDIF %prt
    \ENDIF %rcmod
  \ELSE %purpcl 1
    \IF {$purpcl \leq 0.001191$}
      \IF {$advmod \leq 0.045825$}
        \STATE $es (6.0)$
      \ELSE %advmod
        \STATE $en (2.0)$
      \ENDIF
    \ELSE %purpcl 2
      \STATE $en (7.0)$
    \ENDIF %purpcl 2
  \ENDIF %purpcl 1
\ELSE
  \IF {$mark \leq 0.00808$}
    \STATE $en (6.0)$
  \ELSE %mark
    \IF {$aux \leq 0.044037$}
      \STATE $en (6.0/1.0)$
    \ELSE %aux
      \STATE $es (60.0)$
    \ENDIF
  \ENDIF
\ENDIF %complm
\end{algorithmic}
}
\label{alg:reln-dtree}
\end{spacing}
\end{algorithm}


\begin{table}
\caption{Relation abbreviations}
\begin{center}
  \begin{tabular}{| l | l |}
    \hline
    advmod & adverbial modifier  \\
    \hline
    aux & auxiliary \\
    \hline
    complm & complementizer \\
    \hline
    mark & marker \\
    \hline
    prt & phrasal verb particle \\
    \hline
    purpcl & purpose clause modifier \\
    \hline
    rcmod & relative clause modifier \\
    \hline
  \end{tabular}
\end{center}
\label{table:reln-abbr}
\end{table}

\subsection{Adverbial Modifier}
\subsection{Auxiliary}
\subsection{Complementizer}

A complementizer is a word that signals the beginning of a clausal complement. The Stanford Parser recognizes the complementizers \textit{that} and \textit{whether} as shown in example~\ref{ex:complm3} (Sulec,Micusp). The governor of a complementizer dependency is the root of the clause, which is generally a verb or, in the cause of copular clauses, the subject complement. The dependent is the complementizer itself.


\eenumsentence{
\label{ex:complm3}
\item
\xytext{
\xybarnode{\ldots} &
\xybarnode{I} &
\xybarnode{will} &
\xybarnode{consider} &
\xybarnode{\ldots} &
\xybarnode{whether} &
\xybarnode{the} &
\xybarnode{world} &
\xybarnode{is} &
\xybarnode{a} &
\xybarnode{safe} &
\xybarnode{place}
\xybarconnect(U,U){-6}"_{\small complm}"
}

\item
\xytext{
\xybarnode{At} &
\xybarnode{least} &
\xybarnode{you} &
\xybarnode{choose} &
\xybarnode{whether} &
\xybarnode{to} &
\xybarnode{go}
\xybarconnect(U,U){-2}"_{\small complm}" &
\xybarnode{to} &
\xybarnode{a} &
\xybarnode{pub} &
\xybarnode{or} &
\xybarnode{not.}
}

\item
\xytext{
\xybarnode{They} &
\xybarnode{state} &
\xybarnode{that} &
\xybarnode{climate} &
\xybarnode{generally} &
\xybarnode{predicts}
\xybarconnect(U,U){-3}"_{\small complm}"&
\xybarnode{that} &
\xybarnode{temperatures} &
\xybarnode{should} &
\xybarnode{rise}
\xybarconnect(U,U){-3}"_{\small complm}"&
\ldots
}
}

\citet{whitley:1986} points out that while English tends to allow complementizers introducing clausal complements in the object position to be deleted, Spanish is much more restrictive in this regard (see examples~\ref{ex:complm1} and~\ref{ex:complm2}). \citep[p.278]{whitley:1986}.

\eenumsentence{
\label{ex:complm1}
\item I say that he'll do it.
\item I say he'll do it.
}

\eenumsentence{
\label{ex:complm2}
\item Digo que lo hará.
\item *Digo lo hará.
}



\subsection{Marker}
\subsection{Phrasal Verb Particle}

The phrasal verb particle relation ties the head word of a phrasal verb to its particle as shown in Example~\ref{ex:prt-en1}. 
\enumsentence{
\xytext{
\xybarnode{\ldots the} &
\xybarnode{reduction} &
\xybarnode{of} &
\xybarnode{superfluous} &
\xybarnode{proteins} &
\xybarnode{will} &
\xybarnode{free}
\xybarconnect(U,U){1}"^{\small prt}"
&
\xybarnode{up} &
\xybarnode{resources \ldots}
}
\label{ex:prt-en1}
}
\subsection{Purpose Clause Modifier}
\subsection{Relative Clause Modifier}

\biblio
\end{document}
