\documentclass[main.tex]{subfiles}
\begin{document}

\section{Conclusion}

Building an accurate classifier to distinguish between native and nonnative texts is very feasible with today's parsing technologies. Even a text lacking syntactic errors can be classified as nonnative through the identification of overused or underused grammatical constructions. This tendency for the learner to use certain grammatical constructions more or less frequently than a native English speaker is likely also part of what make nonnative texts appear as such to native readers, though human readers undoubtedly take semantics into consideration as well when making such a determination. As shown in the previous chapter, it is feasible to develop a tool based on the classification principles shown here which would help a learner to become aware of the features of his or her writing that identify it as nonnative.

This study considered three approaches to gathering grammatical features for use in the classification process. The first of these used grammatical relations of all types generated by the Stanford Parser. By the nature of grammatical relations, this approach looked at a broad range of grammatical constructions, including such diverse elements as the use of the inflected genitive and the use of phrasal verbs. This breadth undoubtedly accounts for its high accuracy. The next two approaches delved more deeply into particular aspects of grammar. The second classification method explored using grammatical dependencies and lexemes to explore the use of pronominal and lexical verbal arguments. The third method extracted information on verbal constructions from parse trees. These three methods are only examples of the types of features sets that can be extracted from text. Other possible feature set could be derived from vocabulary (perhaps focusing on closed class words), syntactic complexity (e.g., the depth of parse trees, nesting of various types of phrases, and so forth), use of contractions, use of indirect objects (e.g., whether prepositions are used or not), the use of copular verbs (e.g., considering words such as \textit{seem}, \textit{become} as compared to \textit{be}), and so forth. The greater the variety of feature sets used, the more useful a learner's tool would be.



\biblio
\end{document}